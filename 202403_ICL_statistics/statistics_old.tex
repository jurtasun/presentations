\documentclass{book}

\usepackage{makeidx}
\makeindex

\begin{document}

\frontmatter

\begin{titlepage}
    \entering
    \vspace*{\fill}
    {\Huge\bfseries Probabiliy theory and statistical inference\par}
    \vspace{1cm}
    {\Large Jesus Urtasun Elizari\par}
    \vspace{1cm}
    {\Large Imperial College London\par}
    \vspace{1cm}
    {\large\today\par}
    \vspace*{\fill}
\end{titlepage}

\tableofcontents

\chapter*{Index}
\addcontentsline{toc}{chapter}{Index}
\printindex

\mainmatter

\chapter*{Introduction}
\addcontentsline{toc}{chapter}{Introduction}
\section{Background}
In the following pages one will find an introductory text to one of the key subjects within mathematical sciences. The text is composed by four chapters, together with some appendix reviewing basic mathematical concepts, and a bibliographic note. The purpose of this lecture notes is to make both probability and statistical analysis an easy, interesting and engaging topic for anyone interested, without the need for prior experience with mathematical training.

First, we will introduce and explore the concept of probability itself, and we will discuss how to model information, surprise, and various random processes, also referred to as stochastic.

\chapter{Introduction to probability theory}
\section{Section 1.1}
Your content for Chapter 1 goes here.

\chapter{Introduction to analysis and linear models}
\section{Section 2.1}
Your content for Chapter 2 goes here.

\chapter{Statistical inference and hypothesis testing}
\section{Section 3.1}
Your content for Chapter 3 goes here.

\chapter{Introduction to bayesian statistics}
\section{Section 4.1}
Your content for Chapter 4 goes here.

\backmatter

\chapter*{Bibliography}
\addcontentsline{toc}{chapter}{Bibliography}
\bibliographystyle{plain} % Choose your bibliography style
\bibliography{your_bibliography_file} % Replace with your actual .bib file

\end{document}